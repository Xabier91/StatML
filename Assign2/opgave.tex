\documentclass{article}

\usepackage{amsmath}
\usepackage{minted}
\usepackage{graphicx}
\usepackage{caption}
\usepackage{subcaption}
\usepackage{mathrsfs}
\usepackage[toc,page]{appendix}

%Section style
\usepackage{etoolbox} %for configuration of sloppy
\usepackage{xcolor}


\definecolor{secnum}{RGB}{102,102,102}

\makeatletter
    \def\@seccntformat#1{\llap{\color{secnum}\csname the#1\endcsname\hskip 16pt}}
\makeatother
%end section style

\begin{document}

\section{II.1}

\subsection{II.1.1}

We implemented the LDA ourselves. The code can be seen in
Part1/Opgavei11.m. 

On the training data we observe a 14.00 \% miss rate while on the test data
the rate is 21.05 \%.

\subsection{II.1.2}

After normalizing the data we observed the same error as on the
non-transformed data. 

This is unsurprising as normalization preserves the relation inbetween
the elements of the dataset. Graphically a plot of the data set would
look identical except for a change of the axises. Finally LDA uses the
covariance which fully describes the distribution of the data and thus
renders normalization redundant.

\subsection{II.1.3}

Muh Bayes.



\end{document}
